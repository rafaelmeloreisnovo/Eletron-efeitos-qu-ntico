\documentclass[aps,prb,reprint,showpacs,superscriptaddress,longbibliography]{revtex4-2}

\usepackage[utf8]{inputenc}
\usepackage[T1]{fontenc}
\usepackage{amsmath,amssymb,bm}
\usepackage{graphicx}
\usepackage{hyperref}
\usepackage{booktabs}

\begin{document}

\title{Evidências Experimentais e Estatísticas de Regime Quântico em Circuitos Elétricos Supercondutores}

\author{Autor Exemplo}
\affiliation{Instituição de Pesquisa}
\date{\today}

\begin{abstract}
Apresenta-se um protocolo acadêmico completo para demonstrar, de modo reprodutível e verificável, a presença de efeitos quânticos em circuitos elétricos baseados em junções Josephson. O método baseia-se em três pilares: (i) observação experimental de assinaturas físicas canônicas do regime quântico; (ii) análise estatística formal comparando modelos térmicos clássicos e de tunelamento quântico macroscópico (MQT); (iii) reprodutibilidade via depósito de dados, código e ambiente computacional.
\end{abstract}

\maketitle

\section{Fundamentos Teóricos}

Considere uma junção Josephson com corrente crítica $I_c$, capacitância $C$ e resistência $R$. A energia de Josephson é
\begin{equation}
E_J = \frac{\hbar I_c}{2e}.
\end{equation}
A frequência de plasma efetiva é dada por
\begin{equation}
\omega_p(I) \approx \sqrt{\frac{2e I_c}{\hbar C}}\;\bigl(1 - (I/I_c)^2\bigr)^{1/4}.
\end{equation}
A altura da barreira de potencial no poço tipo ``washboard'' é
\begin{equation}
\Delta U(I) \approx \frac{2\sqrt{2}}{3} E_J \bigl(1 - I/I_c \bigr)^{3/2}.
\end{equation}

\subsection{Taxas de escape}

Para ativação térmica (TA),
\begin{equation}
\Gamma_{\mathrm{TA}}(T) \propto \omega_p \exp\left(-\frac{\Delta U}{k_BT}\right).
\end{equation}
Para tunelamento quântico macroscópico (MQT) em baixa temperatura,
\begin{equation}
\Gamma_{\mathrm{MQT}} \approx a\,\omega_p \exp\left[-\frac{7.2\,\Delta U}{\hbar\,\omega_p}\left(1+\frac{0.87}{Q}\right)\right],
\end{equation}
onde $Q$ é o fator de qualidade efetivo do poço.

A assinatura chave é a saturação da taxa de escape $\Gamma(T)$ para $T \to 0$, incompatível com ativação térmica.

\section{Métodos Experimentais}

\subsection{Montagem}
\begin{itemize}
\item Amostras: junções Josephson caracterizadas quanto a $I_c$, $R$, $C$.
\item Criogenia: $^3$He ou diluição, estabilidade de temperatura $<1$ mK.
\item Blindagem: $\mu$-metal + cobre OFHC; filtros RC e $\pi$ em linhas.
\item Medição: rampa de corrente controlada; detecção de switching; ADC sincronizado.
\item Micro-ondas: gerador sintetizado com acoplamento calibrado.
\end{itemize}

\subsection{Protocolo}
\begin{enumerate}
\item Medir a distribuição de correntes de escape $P(I_\mathrm{sw})$ para diferentes temperaturas.
\item Ajustar os dados a $\Gamma_{\mathrm{TA}}(T)$ e $\Gamma_{\mathrm{MQT}}$.
\item Realizar espectroscopia de micro-ondas e identificar ressonâncias discretas.
\item Executar controles negativos e experimentos de ablação.
\end{enumerate}

\section{Análise Estatística}

\subsection{Ajuste de Modelos}
Os parâmetros $\omega_p$, $\Delta U$, $Q$ são extraídos por máxima verossimilhança com intervalos de confiança de 95\%. São comparados os modelos TA (clássico) e MQT (quântico) por razão de verossimilhança (LRT) e fator de Bayes $BF_{Q/C}$.

\subsection{Critérios de Evidência}
\begin{itemize}
\item Saturação de $\Gamma(T)$ em baixa $T$ incompatível com TA.
\item $BF_{Q/C} > 10$ indicando forte preferência pelo modelo quântico.
\item Presença de níveis discretos e coerentes em espectroscopia.
\item Controles negativos sem assinaturas quânticas.
\end{itemize}

\section{Reprodutibilidade}

\subsection{Depósito de Dados}
Todos os dados brutos, scripts de análise e ambientes computacionais (conda/Docker) são publicados com DOI e hashes SHA-256. A estrutura recomendada segue padrões reprodutíveis:

\begin{verbatim}
PROJECT_ROOT/
  CODE/
  DATA/RAW/
  DATA/PROC/
  ANALYSIS/
  RESULTS/
  DOCS/
  ENV/
  PROVENANCE/
\end{verbatim}

\section{Conclusão}

O protocolo apresentado fornece base experimental e estatística sólida para distinguir regimes quânticos e clássicos em circuitos supercondutores, de forma verificável e replicável por terceiros.

\bibliographystyle{apsrev4-2}
\begin{thebibliography}{99}

\bibitem{devoret} M. H. Devoret, J. M. Martinis, e J. Clarke, Phys. Rev. Lett. \textbf{55}, 1908 (1985).

\bibitem{martinis1987} J. M. Martinis, M. H. Devoret, J. Clarke, Phys. Rev. B \textbf{35}, 4682 (1987).

\bibitem{tinkham} M. Tinkham, \textit{Introduction to Superconductivity}, 2nd ed., McGraw–Hill (1996).

\bibitem{nakamura1999} Y. Nakamura, Y. Pashkin, J. Tsai, Nature \textbf{398}, 786 (1999).

\end{thebibliography}

\end{document}
